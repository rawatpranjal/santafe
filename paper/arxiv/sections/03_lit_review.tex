The study of price formation in decentralized markets has evolved through a dialectic between theoretical pessimism and empirical optimism. Early work focused on the impossibility of equilibrium without a central auctioneer, a view overturned by experimental evidence demonstrating the remarkable robustness of the Double Auction (DA) institution. This section traces the intellectual history from the first classroom experiments to the computational tournaments that set the stage for modern algorithmic trading.

\subsection{Early Experimental Markets: From Chaos to Equilibrium}

The experimental investigation of market behavior began with \citet{chamberlin1948}, who sought to test the neoclassical theory of competitive equilibrium in a controlled setting. Chamberlin's design involved a decentralized bilateral bargaining process where students, acting as buyers and sellers with private reservation values, roamed a room to negotiate trades. Chamberlin observed that transaction prices fluctuated widely and the quantity traded consistently exceeded the competitive equilibrium prediction. He concluded that decentralized markets were inherently imperfect and that the theoretical intersection of supply and demand was an abstraction unlikely to be realized in practice without a mechanism for recontracting.

This conclusion was challenged and ultimately reversed by \citet{smith1962}. Smith hypothesized that the inefficiency observed by Chamberlin was not due to the bounded rationality of the agents, but to the unstructured nature of bilateral bargaining. Smith introduced a centralized public clearing mechanism—the oral Double Auction—where bids and asks were announced to the entire market, and transactions occurred at publicly known prices. Under these rules, Smith observed rapid convergence to the competitive equilibrium price and quantity, often reaching allocative efficiencies exceeding 95\% within a few trading periods. Crucially, this convergence occurred despite agents possessing only private information and no knowledge of the aggregate supply and demand schedules. Smith's findings validated the Hayekian hypothesis that market institutions serve as information aggregators \citep{hayek1945}, demonstrating that the structure of the institution is a primary determinant of market efficiency. Notably, Smith observed that convergence followed a predictable directional path: in markets with excess supply, prices tended to start high and glide downward, while in markets with excess demand, prices started low and rose toward equilibrium, a pattern later confirmed by \citet{cliff1997}.

\subsection{Theoretical Foundations: Heuristics and Game Theory}

Following Smith's empirical success, theorists sought to explain \textit{why} the Double Auction converges so reliably. Two distinct approaches emerged: game-theoretic equilibrium analysis and behavioral learning models.

\citet{wilson1987} provided the first rigorous game-theoretic treatment of the DA with incomplete information. Modeling the market as a multilateral sequential bargaining game, Wilson derived a sequential equilibrium in which traders adopt a "waiting game" strategy. Sellers with high costs and buyers with low valuations wait to reveal their offers, using delay as a credible signal of their private information. Wilson showed that as the number of traders increases, this strategic delay diminishes, and the market outcome converges asymptotically to the Walrasian equilibrium. However, Wilson’s model relies on strong assumptions of common knowledge and sophisticated rationality that are difficult to justify in human subjects or simple software agents.

In contrast, \citet{easley1993} proposed a behavioral model based on simple adaptive heuristics. They assumed that traders do not optimize against the entire market state but instead adjust their "reservation prices"—mental thresholds for bidding and asking—based on past success or failure. A trader who fails to transact becomes more aggressive (raising bids or lowering asks) in the next period, while a trader who transacts easily becomes more passive. Easley and Ledyard proved that these simple learning dynamics are sufficient to trap transaction prices within a corridor that converges to the competitive equilibrium, providing a robust explanation for Smith's results that does not require hyper-rationality.

Bridging these approaches, \citet{friedman1991} modeled the DA as a "Game Against Nature," where a rational trader treats the arrival of bids and asks as a stochastic process rather than the strategic output of opponents. Under this framework, Friedman derived an optimal "aggressive reservation price" strategy, where traders shade their bids to maximize expected surplus. This strategy mathematically resembles the "sniping" behavior predicted by Wilson's waiting game, suggesting a convergence between optimal control and game-theoretic predictions.

\subsection{Zero-Intelligence and the The Santa Fe Tournament}

The role of agent intelligence was radically questioned by \citet{gode1993} in their seminal work on "Zero-Intelligence" (ZI) traders. They simulated a DA market populated by algorithmic agents that submitted random bids and asks subject only to a budget constraint (ZI-C agents could not buy above their valuation or sell below cost). Surprisingly, these random agents achieved allocative efficiencies close to 100\%, statistically indistinguishable from human markets. This finding, dubbed the "Zero-Intelligence" result, implied that the allocative efficiency of the DA is largely an emergent property of the market rules (specifically the budget constraint and the public order book) rather than a product of trader learning or strategy. However, while ZI-C agents maximized the total market surplus, Gode and Sunder also found that profit dispersion among individual ZI traders was enormous compared to human markets; some agents earned far above their theoretical share while others earned almost nothing, suggesting that while market structure ensures allocative efficiency, individual intelligence is required to secure an equitable distribution of gains.

However, while ZI agents achieved high \textit{efficiency}, they failed to extract surplus strategically. To investigate the limits of algorithmic trading, the Santa Fe Institute organized a Double Auction Tournament in 1990 \citep{rust1993, rust1994}. The tournament invited researchers to submit trading programs to compete in a synchronized discrete-time DA. The results were striking: simple heuristic strategies consistently outperformed complex learning algorithms (such as early neural networks). The tournament was won by the "Kaplan" strategy \citep{kaplan1993}, a simple sniper that waited in the background until the bid-ask spread narrowed before jumping in to steal the deal.

The computational investigation of market microstructure reached a watershed moment with the Santa Fe Double Auction Tournament, organized by \citet{rust1993, rust1994}. Moving beyond the representative agent paradigm, the organizers invited researchers to submit diverse trading algorithms to compete in a "Synchronized Double Auction"—a discrete-time approximation of the continuous market where agents simultaneously submit limit orders, followed by a trade execution phase governed by AURORA rules. The tournament field was highly heterogeneous, featuring strategies ranging from simple rule-based heuristics to sophisticated neural networks and genetic algorithms. Contrary to the expectations of the artificial intelligence community, the tournament was not won by a complex learning agent, but by a simple heuristic strategy submitted by Todd Kaplan. The "Kaplan" strategy functioned as a sniper: it remained passive in the background, observing the bid-ask spread, and only entered the market to "steal the deal" when the spread narrowed sufficiently or the trading period neared its conclusion. This parasitic strategy exploited the information revealed by more impatient traders (such as ZI-C or GD agents) while minimizing its own exposure to the winner's curse. However, \citeauthor{rust1994} engaged in a subsequent evolutionary analysis that revealed a profound paradox: while Kaplan agents dominated heterogeneous populations, a market composed entirely of Kaplan agents collapsed into a state of liquidity failure. With every agent waiting to snipe an offer that never materialized, transaction volume plummeted and allocative efficiency fell to approximately 50\%. This "Kaplan deadlock" demonstrated that while sniping is locally optimal for an individual in a liquid market, it is globally unstable as a dominant strategy, underscoring the necessity of "noise traders" or impatience to lubricate the mechanism of price discovery.

Rust et al. noted a critical fragility in the Kaplan strategy: while it dominated heterogeneous markets, a market composed entirely of Kaplan agents collapsed. Since every agent waited for another to provide liquidity, trading volume plummeted, and efficiency dropped to approximately 50\%. Beyond this liquidity failure, Rust, Palmer, and Miller decomposed the sources of inefficiency in agent-based markets and identified a phenomenon they termed extra-marginal displacement: aggressive traders with unfavorable cost or value positions could "steal" trades from more efficient intra-marginal traders, a dynamic distortion missed by static equilibrium theory. This "Kaplan deadlock" highlighted a fundamental tension between individual rationality (sniping) and collective efficiency (liquidity provision), a problem that remains central to the study of automated market makers and algorithmic trading today.

\subsection{The Post-Tournament Era: Adaptation and Optimization}

In the wake of the Santa Fe tournament, researchers sought to dissect why simple heuristics succeeded where complex models failed, and to push the boundaries of algorithmic performance. \citet{cason1996} conducted a rigorous laboratory investigation to test three competing theoretical frameworks—Wilson’s waiting game, Friedman’s Bayesian model, and Gode and Sunder’s zero-intelligence hypothesis—against human behavior. Crucially, they introduced a "random valuation" environment where trader values change every period, preventing the simple rote learning of a static equilibrium price. Their results confirmed that while zero-intelligence agents could explain the baseline efficiency of the Double Auction, they failed to capture the dynamics of transaction order and bid progressions observed in experienced human traders. As humans gained experience, their behavior shifted away from randomness toward the strategic patterns predicted by Bayesian models, suggesting that market efficiency is not merely a structural artifact but also a product of learning. This validates the use of learning algorithms like PPO in random-valuation environments, as they mimic the adaptive trajectory of human subjects.

Responding to the limitations of zero-intelligence, \citet{cliff1997} introduced the "Zero-Intelligence Plus" (ZIP) agent. Cliff and Bruten demonstrated that while ZI-C agents achieve high efficiency in symmetric markets, they fail to converge to equilibrium prices in markets with asymmetric supply and demand schedules. To bridge this gap, they endowed agents with a simple adaptive mechanism based on the Widrow-Hoff delta rule. ZIP agents maintain a profit margin that they adjust heuristically: lowering margins to remain competitive when trades are scarce, and raising them to extract surplus when trades are frequent. This minimal adaptivity was sufficient to produce human-like price convergence in complex markets where ZI-C failed. For our research, ZIP represents a critical benchmark: a "behavioral" agent that learns scalar parameters (margins) rather than a full policy, providing a middle ground between random noise and deep reinforcement learning.

While ZIP focused on heuristics, \citet{gjerstad1998} returned to the principles of optimization. They developed the "GD" strategy, which constructs a belief function estimating the probability that any given bid or ask will be accepted based on the recent history of market orders and transactions. The GD agent then chooses a price that maximizes its expected surplus against this belief function. In simulations, GD agents achieved near-perfect efficiency and converged to equilibrium faster than human subjects, establishing a new standard for algorithmic performance. The success of GD highlights the value of market history—specifically the order book and transaction log—as a state representation. This suggests that for a PPO agent to compete with or outperform GD, its observation space must include sufficiently rich historical features to implicitly reconstruct similar belief functions.

The algorithmic arms race continued with \citet{tesauro2001}, who introduced a modified version of the GD algorithm (MGD) and tested it in a realistic, continuous-time environment. They found that the original GD strategy could be volatile and unstable in certain market conditions. By adding heuristic stabilizations and extending the belief-based approach to handle persistent orders, their MGD strategy consistently outperformed both ZIP and the original Kaplan sniping strategy in head-to-head tournaments. \citeauthor{tesauro2001} demonstrated that while sniping (Kaplan) exploits naive agents effectively, it is vulnerable to sophisticated belief-based agents that can optimize their pricing dynamically. This finding poses a direct challenge to our PPO agents: to claim state-of-the-art performance, they must not only rediscover sniping but also demonstrate robustness against optimized belief-based strategies like MGD.

Finally, the potential for evolutionary discovery in these markets was explored by \citet{chen2010} and \citet{chen2011} using Genetic Programming (GP). Unlike previous approaches that hand-coded strategies, they allowed agents to evolve trading rules from basic mathematical and logical primitives. Their GP agents eventually discovered sophisticated "optimal procrastination" strategies that mirrored the Kaplan sniper but with greater adaptability to market shape. By analyzing the syntactic trees of the evolved agents, they found that the agents had learned to assess their competitive position and exercise monopsony power by withholding bids until the optimal moment. This confirms that the "sniping" behavior is not an artifact of a specific heuristic but a fundamental attractor in the strategy space of the Double Auction—one that we expect Deep Reinforcement Learning to rediscover and perhaps refine through gradient-based optimization.