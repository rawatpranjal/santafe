\section{Outcome Metrics for Continuous Double Auctions}
\label{app:metrics}

This appendix provides formal definitions for all outcome metrics used to evaluate agent and market performance in this study. The notation follows the Santa Fe Double Auction Tournament \citep{rust1994}, with metric definitions drawn from \citet{gode1993}, \citet{cliff1997}, \citet{gjerstad1998}, \citet{chen2010}, and \citet{smith1962}.

\subsection{Mathematical Notation and Preliminaries}

\subsubsection{The Environment}

Let $B$ denote the set of buyers and $S$ denote the set of sellers participating in the market. Each buyer $i \in B$ holds a sequence of units with redemption values $v_{i1}, v_{i2}, \ldots$, where $v_{ik}$ represents the value of buyer $i$'s $k$-th unit. Similarly, each seller $j \in S$ holds units with costs $c_{j1}, c_{j2}, \ldots$, where $c_{jk}$ represents the cost of seller $j$'s $k$-th unit. Table \ref{tab:environment_notation} summarizes this notation.

\begin{table}[h]
\centering
\caption{Environment Notation}
\label{tab:environment_notation}
\begin{tabular}{ll}
\toprule
Symbol & Definition \\
\midrule
$B$ & Set of buyers \\
$S$ & Set of sellers \\
$v_{ik}$ & Redemption value for buyer $i$'s $k$-th unit \\
$c_{jk}$ & Cost for seller $j$'s $k$-th unit \\
\bottomrule
\end{tabular}
\end{table}

\subsubsection{Demand and Supply Schedules}

The aggregate demand schedule $D(q)$ is constructed by ordering all buyer valuations $v_{ik}$ in descending order. The value $D(q)$ represents the redemption value of the $q$-th unit on the aggregated demand curve. The aggregate supply schedule $S(q)$ is constructed by ordering all seller costs $c_{jk}$ in ascending order. The value $S(q)$ represents the cost of the $q$-th unit on the aggregated supply curve.

\subsubsection{Equilibrium Definitions}

The equilibrium quantity $Q^*$ is defined as the maximum quantity where demand exceeds supply:
\begin{equation}
Q^* = \max\{q : D(q) > S(q)\}
\end{equation}

The equilibrium price $P^*$ is any price in the marginal interval bounded by the marginal demand and supply:
\begin{equation}
S(Q^*) \leq P^* \leq D(Q^*)
\end{equation}
In practice, $P^*$ is often defined as the midpoint: $P^* = (D(Q^*) + S(Q^*))/2$.

The maximum theoretical surplus $TS^*$ represents the total gains from trade available if all profitable exchanges occur:
\begin{equation}
TS^* = \sum_{q=1}^{Q^*} \bigl(D(q) - S(q)\bigr)
\end{equation}

\subsubsection{Market Activity Notation}

Let $t = 1, \ldots, T$ index the sequence of concluded transactions within a trading period. For each transaction $t$, let $p_t$ denote the transaction price, $v_t$ denote the redemption value of the unit exchanged, and $c_t$ denote the cost of the unit exchanged. Table \ref{tab:activity_notation} summarizes this notation.

\begin{table}[h]
\centering
\caption{Market Activity Notation}
\label{tab:activity_notation}
\begin{tabular}{ll}
\toprule
Symbol & Definition \\
\midrule
$t = 1, \ldots, T$ & Sequence of concluded transactions \\
$p_t$ & Transaction price at trade $t$ \\
$v_t$ & Redemption value of unit exchanged at trade $t$ \\
$c_t$ & Cost of unit exchanged at trade $t$ \\
\bottomrule
\end{tabular}
\end{table}

\subsection{Market Efficiency Metrics}

These metrics evaluate the aggregate performance of the market in extracting potential gains from trade.

\subsubsection{Allocative Efficiency}

The primary efficiency metric from \citet{smith1962} and \citet{gode1993} measures the percentage of maximum possible surplus actually realized:
\begin{equation}
E = \frac{\sum_{t=1}^{T} (v_t - c_t)}{TS^*} \times 100
\end{equation}

If traders exchange units where $c_t > v_t$ (negative surplus trades), the numerator decreases, lowering efficiency. Table \ref{tab:efficiency_benchmarks} provides benchmark values from the literature.

\begin{table}[h]
\centering
\caption{Allocative Efficiency Benchmarks}
\label{tab:efficiency_benchmarks}
\begin{tabular}{lll}
\toprule
Trader Type & Expected $E$ & Reference \\
\midrule
ZI (unconstrained) & 60-70\% & Gode \& Sunder 1993 \\
ZIC (constrained) & 98.7\% & Gode \& Sunder 1993 \\
ZIP & 99.9\% & Cliff \& Bruten 1997 \\
GD & $>$99.9\% & Gjerstad \& Dickhaut 1998 \\
Mixed tournament & 89.7\% & Rust et al.\ 1994 \\
\bottomrule
\end{tabular}
\end{table}

\subsubsection{Efficiency Loss Decomposition}

Following \citet{rust1994}, the total lost surplus $(100\% - E)$ can be decomposed into four components. Define intra-marginal units as those that should trade ($q \leq Q^*$) and extra-marginal units as those that should not trade ($q > Q^*$).

Intra-marginal loss (IM) represents surplus lost from failing to trade profitable units:
\begin{equation}
IM = \sum_{q \in \text{Untraded Intra-marginal}} \bigl(D(q) - S(q)\bigr)
\end{equation}

Extra-marginal loss (EM) represents negative surplus from trading units that should not have been traded:
\begin{equation}
EM = \sum_{t \in \text{Extra-marginal trades}} (c_t - v_t)
\end{equation}

Buyer displacement (BS) captures surplus lost when an extra-marginal buyer displaces an intra-marginal buyer. Seller displacement (SS) captures surplus lost when an extra-marginal seller displaces an intra-marginal seller.

The decomposition identity states:
\begin{equation}
100\% - E = IM + EM + BS + SS
\end{equation}

\subsection{Price Convergence Metrics}

These metrics measure the tendency of transaction prices to approach the equilibrium price $P^*$.

\subsubsection{Root Mean Squared Deviation}

Following \citet{gode1993}, the RMSD measures the distance of prices from equilibrium:
\begin{equation}
RMSD = \sqrt{\frac{1}{T} \sum_{t=1}^{T} (p_t - P^*)^2}
\end{equation}

\subsubsection{Smith's Alpha}

The coefficient of convergence from \citet{smith1962} normalizes the standard deviation of prices around equilibrium by the equilibrium price. Let $\sigma_0$ be the root mean squared deviation of prices around equilibrium:
\begin{equation}
\sigma_0 = \sqrt{\frac{1}{T} \sum_{t=1}^{T} (p_t - P^*)^2}
\end{equation}

Smith's alpha is then:
\begin{equation}
\alpha = \frac{100 \cdot \sigma_0}{P^*}
\end{equation}

Lower values of $\alpha$ indicate tighter convergence to equilibrium. Some sources use a scaling factor of 1000 instead of 100, though the interpretation remains the same.

\subsubsection{Price Standard Deviation}

The raw volatility measure captures dispersion around the mean transaction price:
\begin{equation}
\sigma_p = \sqrt{\frac{1}{T} \sum_{t=1}^{T} (p_t - \bar{p})^2}
\end{equation}
where $\bar{p} = (1/T)\sum_{t=1}^{T} p_t$ is the mean transaction price. ZIC traders exhibit high, constant volatility (2-3 times human levels), while ZIP and GD traders show declining volatility as they learn.

\subsubsection{Price Volatility Percentage}

For cross-market comparison, volatility can be normalized:
\begin{equation}
\text{Volatility\%} = \frac{\sigma_p}{\bar{p}} \times 100
\end{equation}
Values below 5\% indicate good convergence, while values above 20\% indicate an unstable market.

\subsubsection{Hit Rate}

From the Santa Fe Tournament, the hit rate measures the percentage of trades within a band around equilibrium:
\begin{equation}
H = \frac{|\{t : |p_t - P^*| \leq 0.05 \cdot P^*\}|}{T}
\end{equation}

\subsubsection{Mean Absolute Deviation}

Following \citet{gjerstad1998}:
\begin{equation}
MAD = \frac{1}{T} \sum_{t=1}^{T} |p_t - P^*|
\end{equation}
ZIP traders typically achieve MAD of approximately \$0.08, while GD traders achieve approximately \$0.04.

\subsection{Trader Performance Metrics}

These metrics evaluate individual agents rather than the market as a whole.

\subsubsection{Individual Profit}

Raw earnings for trader $i$ are computed as follows. For a buyer:
\begin{equation}
\pi_i = \sum_{k \in \text{Items Traded}} (v_{ik} - p_k)
\end{equation}

For a seller:
\begin{equation}
\pi_j = \sum_{k \in \text{Items Traded}} (p_k - c_{jk})
\end{equation}

\subsubsection{Equilibrium Profit}

The theoretical profit at competitive equilibrium represents what trader $i$ would earn if all trades occurred at $P^*$. For a buyer:
\begin{equation}
\pi_i^* = \sum_{k : v_{ik} > P^*} (v_{ik} - P^*)
\end{equation}

For a seller:
\begin{equation}
\pi_j^* = \sum_{k : c_{jk} < P^*} (P^* - c_{jk})
\end{equation}

\subsubsection{Profit Deviation}

The difference between actual and equilibrium profit indicates whether a trader extracted more or less than their fair share:
\begin{equation}
\Delta\pi_i = \pi_i - \pi_i^*
\end{equation}

Positive values indicate the trader extracted more than fair share, zero indicates exactly fair share, and negative values indicate underperformance or exploitation.

\subsubsection{Individual Efficiency Ratio}

Following \citet{chen2010}, the ratio of actual to theoretical profit:
\begin{equation}
E_i = \frac{\pi_i}{\pi_i^*}
\end{equation}

Values greater than 1 indicate the trader captures more than their equilibrium share (exploiter), values equal to 1 indicate exactly equilibrium share, and values less than 1 indicate the trader is being exploited. Table \ref{tab:efficiency_ratio_benchmarks} provides benchmark values.

\begin{table}[h]
\centering
\caption{Individual Efficiency Ratio Benchmarks}
\label{tab:efficiency_ratio_benchmarks}
\begin{tabular}{ll}
\toprule
Trader Type & Expected $E_i$ \\
\midrule
Kaplan (mixed market) & 1.14-1.21 \\
ZIC & $\approx$1.0 \\
ZIP/GD & $\approx$1.0 \\
Kaplan (pure market) & 0.5-0.6 \\
\bottomrule
\end{tabular}
\end{table}

\subsubsection{Profit Dispersion}

This metric from \citet{cliff1997} is the key metric for discriminating intelligent from zero-intelligence traders. It measures the cross-sectional RMS difference between actual and equilibrium profits:
\begin{equation}
PD = \sqrt{\frac{1}{N} \sum_{i=1}^{N} (\pi_i - \pi_i^*)^2}
\end{equation}
where $N$ is the total number of traders.

ZIC traders exhibit profit dispersion values of 0.35-0.60, reflecting random surplus allocation. ZIP traders achieve approximately 0.05 after convergence, demonstrating that fair allocation emerges from learning. ZIP achieves 7-10 times lower dispersion than ZIC. Even with similar allocative efficiency, profit dispersion reveals whether the ``right'' traders are earning profits.

\subsubsection{Number of Trades}

The activity level for agent $i$:
\begin{equation}
N_i = |\{t : \text{agent } i \text{ participated in trade } t\}|
\end{equation}
Kaplan typically has fewer trades than ZIC due to its waiting strategy.

\subsection{Inequality and Distributional Metrics}

Unlike standard income-inequality applications, profits in double auctions can be negative. This makes some inequality metrics (especially Gini) unstable when mean profit is close to zero or when large losses offset large gains. We therefore supplement the Gini coefficient with ratio-based metrics that remain interpretable across all market configurations.

\subsubsection{Gini Coefficient}

The Gini coefficient measures profit concentration across traders:
\begin{equation}
G = \frac{\sum_{i=1}^{N} \sum_{j=1}^{N} |\pi_i - \pi_j|}{2N \sum_{i=1}^{N} \pi_i}
\end{equation}
where $\pi_i$ is the profit of trader $i$. Values range from 0 (perfect equality) to 1 (one agent captures all profit). The Gini coefficient is well behaved when $\mu > 0$ and all $\pi_i$ are non-negative; when profits cross zero, its interpretation becomes less straightforward.

\subsubsection{Max/Mean Ratio}

The ratio of the highest individual profit to the mean profit:
\begin{equation}
\text{Max/Mean} = \frac{\max_i(\pi_i)}{\bar{\pi}}
\end{equation}
where $\bar{\pi} = (1/N)\sum_i \pi_i$. This metric is most interpretable when the mean is comfortably above zero; when $\bar{\pi}$ is very small, the ratio can be arbitrarily large. Values near 1 indicate equality; values above 2 suggest the presence of ``superstar'' traders.

\subsubsection{Bottom-50\% Share}

The fraction of total profit captured by the lower half of the profit distribution:
\begin{equation}
\text{Bottom-50\%} = \frac{\sum_{i \in \text{bottom half}} \pi_i}{\sum_{i=1}^{N} \pi_i}
\end{equation}
With perfect equality, this equals 50\%. Values below 50\% indicate concentration at the top. Negative values indicate the bottom half of traders collectively lost money.

\subsubsection{Skewness}

The third standardized moment of the profit distribution:
\begin{equation}
\gamma = \frac{E[(\pi_i - \bar{\pi})^3]}{\sigma_\pi^3}
\end{equation}
Positive skewness indicates a long right tail (a few large winners), while negative skewness indicates a long left tail (a few large losers).

\subsection{Dynamic Metrics}

\subsubsection{Price Autocorrelation}

This metric tests whether price changes predict subsequent changes:
\begin{equation}
\rho = \text{Corr}(\Delta p_t, \Delta p_{t-1})
\end{equation}
where $\Delta p_t = p_t - p_{t-1}$.

Negative values indicate mean-reversion where prices overshoot then correct. Zero indicates a random walk with no predictability. Positive values indicate momentum or trending. Empirically, $\rho \approx -0.25$ was found by \citet{rust1994}, rejecting Wilson's (1987) martingale prediction of $\rho = 0$.

\subsubsection{Gode-Sunder Convergence Coefficient}

From \citet{gode1993}, this metric tests whether the market learns within a period. Let $y_t$ be the root mean squared deviation of transaction prices at sequence number $t$, calculated across $N$ experimental runs:
\begin{equation}
y_t = \sqrt{\frac{1}{N} \sum_{n=1}^{N} (p_{t,n} - P^*)^2}
\end{equation}

Regress $y_t$ against $t$:
\begin{equation}
y_t = \alpha + \beta \cdot t + \epsilon_t
\end{equation}

Negative $\beta$ indicates the market is converging (variance shrinking), while $\beta \approx 0$ indicates the market is stagnant (common in ZI unconstrained). The regression is performed on ensemble RMSD across multiple runs, not single-run squared error, to reduce noise.

\subsubsection{Convergence Time}

The number of periods until prices stabilize within a tolerance of equilibrium:
\begin{equation}
T^* = \min\{t : |p_t - P^*| \leq 0.05 \cdot P^*\}
\end{equation}

GD typically achieves $T^* < 1$ period, ZIP requires 1-2 periods, and ZIC never converges due to absence of learning.

\subsubsection{Time of Last Transaction}

From \citet{rust1994}, this metric measures liquidity risk and closing panics:
\begin{equation}
T_{last} = \max_t(\tau_t)
\end{equation}
where $\tau_t$ is the timestamp of trade $t$ and $T_{max}$ is maximum time allowed.

If $T_{last} \approx T_{max}$ consistently, this indicates ``wait in background'' strategies (like Kaplan) causing deadline congestion.

\subsubsection{Rank Correlation of Efficient Order}

This metric measures whether the ``right'' trades happened in the ``right'' order. Theory suggests the highest-value buyer should trade with the lowest-cost seller first. Let $R_{actual}$ be the rank vector of trades by surplus as they occurred and $R_{ideal}$ be the rank vector sorted by theoretical surplus. Then:
\begin{equation}
\rho_s = \text{Spearman}(R_{actual}, R_{ideal})
\end{equation}

A value of $\rho_s = 1.0$ means the market perfectly executed the most profitable trades first.

\subsection{Evolutionary Metrics}

For long-run tournament analysis following \citet{rust1994} and \citet{chen2010}.

\subsubsection{Capital Stock Evolution}

The market share of strategy $i$ at game or generation $g$:
\begin{equation}
K_{i,g} = K_{i,g-1} + \pi_{i,g} - S_{i,g}
\end{equation}
where $S_{i,g}$ is the theoretical surplus assigned to trader $i$.

Strategies with $K$ trending upward are evolutionarily stable; those trending to zero are eliminated.

\subsubsection{Generations to Convergence}

From \citet{chen2010}, this learning speed metric is defined as:
\begin{equation}
Gen^* = \min\{g : E_{pop,g} \geq E_{target}\}
\end{equation}
where $E_{pop,g}$ is the average efficiency at generation $g$ and $E_{target}$ is a threshold (e.g., 99\%).

\subsection{Microstructure Metrics}

\subsubsection{Initiator Price Bias}

From \citet{gjerstad1998}, this metric measures the difference between buyer-initiated and seller-initiated trade prices. Let $T_{buy}$ denote trades where the buyer accepted the standing ask, and $T_{sell}$ denote trades where the seller accepted the standing bid.

\begin{equation}
\bar{p}_{buy} = \frac{1}{|T_{buy}|} \sum_{t \in T_{buy}} p_t
\end{equation}

\begin{equation}
\bar{p}_{sell} = \frac{1}{|T_{sell}|} \sum_{t \in T_{sell}} p_t
\end{equation}

\begin{equation}
\Delta_{init} = \bar{p}_{sell} - \bar{p}_{buy}
\end{equation}

In human markets, $\Delta_{init} \neq 0$ indicates asymmetric urgency between buyers and sellers.

\subsubsection{ZIP Margin Adjustment}

From \citet{cliff1997}, the learning dynamics of the profit margin $\mu$:
\begin{equation}
\Delta\mu_i(t) = \beta \cdot (Target_i(t) - p_i(t))
\end{equation}
where $\beta$ is the learning rate parameter.

The optimal $\beta$ that matches human data becomes an outcome when calibrating agent behavior to empirical markets.
