\section{The Santa Fe Double Auction}

This section revisits foundational questions posed by the original Santa Fe Double Auction Tournament (Rust et al., 1994) and subsequent analyses (Chen & Tai, 2010). Our objective is to investigate the ecological dynamics of a diverse set of heuristic-based trading strategies, examining the mechanisms behind their competitive success or failure within a complex market ecosystem. Our analysis proceeds by first examining easy play scenarios, then homogeneous self-play, and finally the dynamics of heterogeneous mixed play.

First, we investigate the competitive performance of simple heuristics, examining the factors contributing to the success of "sniper" strategies such as Kaplan and Ringuette in heterogeneous market environments. Second, we examine market stability in homogeneous populations, testing whether opportunistic strategies lead to collective market instability and liquidity collapse when agents interact only with copies of themselves, as hypothesized by Rust et al. Third, we analyze the relationship between strategic complexity and ecological fitness, comparing fixed-rule agents against more cognitively elaborate models to understand the trade-offs involved.

Our experimental design involves three distinct setups across the ten Santa Fe market environments: an Invasibility experiment (1 challenger vs. 7 ZIC agents) to assess exploitative capabilities, a Homogeneous Self-Play experiment (8 identical agents) to evaluate collective stability, and a Heterogeneous Round-Robin Tournament (all agents competing) to gauge overall ecological fitness. The agent roster for these experiments includes ZIC (baseline), Skeleton (simple fixed-rule), Kaplan (original sniper), Ringuette (alternative sniper), EL (Ledyard, reservation price model), BGAN (belief-based), Staecker, Gamer, Jacobson, Perry (adaptive parameters), Lin (statistical), and Breton.

\begin{figure}[htbp]
    \centering
    \includegraphics[width=\textwidth]{figures/santafe_selfplay_dynamics.pdf}
    \caption{Self-play trading dynamics for all 13 Santa Fe strategies in the BASE environment. Each panel shows bid submissions (blue triangles), ask submissions (red triangles), best bid/ask trajectories (step lines), and executed trades (large colored circles). The green dashed line indicates the competitive equilibrium price. Ringuette and Kaplan (snipers) achieve high efficiency through patient opportunism, while Skeleton and Breton (fixed-rule) maintain steady trading activity. ZIP (adaptive learning) and Perry (heuristic) show distinct trading patterns reflecting their strategic approaches.}
    \label{fig:s6_selfplay_dynamics}
\end{figure}

\subsection{Easy Play Performance (Invasibility Experiment)}

Our investigation into easy play scenarios focuses on the "Invasibility Experiment," where individual Santa Fe strategies compete against a homogeneous population of baseline ZIC agents. This setup quantifies the exploitative capabilities of each strategy, measuring their ability to extract surplus from a less sophisticated, liquidity-providing market. Table~\ref{table:s6_easy_play_profit_ratio} presents the profit ratio for each Santa Fe agent when playing against ZIC in the BASE environment, indicating how many times more profit they generate compared to a ZIC agent in the same scenario.

\begin{table}[htbp]
    \centering
    \caption{Easy Play Profit Ratios (vs. ZIC, BASE Environment)}
    \label{table:s6_easy_play_profit_ratio}
    \begin{tabular}{lr}
        \toprule
        Strategy & Profit Ratio vs. ZIC \\
        \midrule
        BGAN & 0.608 \\
        Breton & 0.811 \\
        EL & 0.978 \\
        Gamer & 0.995 \\
        Jacobson & 0.961 \\
        Kaplan & 0.844 \\
        Lin & 0.876 \\
        Perry & 1.005 \\
        Ringuette & 0.970 \\
        Skeleton & 0.839 \\
        Staecker & 0.837 \\
        ZIC & 1.000 \\
        \bottomrule
    \end{tabular}
    \caption*{A profit ratio greater than 1 indicates the strategy extracts more profit than a ZIC agent would in the same scenario.}
\end{table}

As Table~\ref{table:s6_easy_play_profit_ratio} demonstrates, several strategies, particularly Perry (1.005), Gamer (0.995), and EL (0.978), exhibit strong exploitative capabilities, generating profits comparable to or exceeding the ZIC baseline when acting as invaders. Ringuette (0.970) and Jacobson (0.961) also show robust performance in this easy play environment. This highlights that strategies capable of adapting to or exploiting simpler market dynamics tend to fare well in asymmetric matchups.

\begin{figure}[htbp]
    \centering
    \includegraphics[width=\textwidth]{figures/santafe_easyplay_dynamics.pdf}
    \caption{Easy-play trading dynamics for all 13 Santa Fe strategies against TruthTeller sellers in the BASE environment. Each panel shows bid submissions (blue triangles), ask submissions (red triangles), best bid/ask trajectories (step lines), and executed trades (large colored circles). The green dashed line indicates the competitive equilibrium price. Most strategies achieve high efficiency when exploiting passive sellers, with the notable exception of Perry (0\% efficiency) which appears to rely on competitive market dynamics that are absent when facing TruthTellers.}
    \label{fig:s6_easyplay_dynamics}
\end{figure}

\subsection{Self Play Performance}
The original Santa Fe paper hypothesized that the success of parasitic strategies like Kaplan might be contingent on a diverse market, and that a market composed solely of such agents could lead to collective instability. Our homogeneous self-play experiments offer substantial evidence supporting the "Sniper's Dilemma" hypothesis.

Table~\ref{table:s6_self_play_metrics_base} provides a summary of key self-play metrics for each Santa Fe strategy in the BASE environment.

\begin{table}[htbp]
    \centering
    \caption{Self Play Metrics (BASE Environment)}
    \label{table:s6_self_play_metrics_base}
    \begin{tabular}{lrrr}
        \toprule
        Strategy & Efficiency ($\%$) & Volatility & Trades/Period \\
        \midrule
        BGAN & 65.54 & 1.73 & 1.01 \\
        Breton & 99.84 & 11.18 & 2.25 \\
        EL & 85.49 & Inf & 1.67 \\
        Gamer & 65.82 & Inf & 1.12 \\
        Jacobson & 40.07 & Inf & 0.84 \\
        Kaplan & 100.00 & 18.38 & 2.00 \\
        Lin & 66.03 & Inf & 1.55 \\
        Perry & 61.59 & Inf & 1.39 \\
        Ringuette & 97.77 & 4.49 & 2.22 \\
        Skeleton & 99.54 & 7.85 & 2.11 \\
        Staecker & 48.75 & Inf & 0.93 \\
        ZIC & 90.80 & Inf & 1.69 \\
        \bottomrule
    \end{tabular}
    \caption*{Efficiency, Volatility, and Trades per Period for each strategy in homogeneous self-play (BASE environment). 'Inf' for volatility indicates insufficient data due to very few or no trades.}
\end{table}

The market efficiency of sniper agents exhibits a substantial decline under specific conditions. In the SHRT (short time) environment, Ringuette's self-play efficiency plummets to 32.5\% (from 98.1\% in BASE), and Kaplan's falls to 79.5\%. This contrasts sharply with the near-perfect efficiency maintained by fixed-rule agents like Skeleton (99.7\%) and Breton (99.8\%) in the same stressful environment.


\begin{table}[htbp]
    \centering
    \caption{Self-Play Profit Dispersion (RMS)}
    \label{table:s6_selfplay_profit_dispersion}
    \begin{tabular}{lrrrrrrrrrr}
        \toprule
        Strategy & BASE & BBBS & BSSS & EQL & RAN & PER & SHRT & TOK & SML & LAD \\
        \midrule
        BGAN & 141 & 75 & 110 & 129 & 849 & 141 & 166 & 15 & 56 & 129 \\
        Breton & 51 & 51 & 51 & 45 & 470 & 51 & 51 & 51 & 51 & 45 \\
        EL & 51 & 89 & 122 & 67 & 68 & 68 & 444 & 163 & 46 & 67 \\
        Gamer & 218 & 218 & 218 & 294 & 580 & 218 & 218 & 218 & 218 & 294 \\
        Jacobson & 571 & 571 & 571 & 634 & 473 & 571 & 571 & 571 & 571 & 634 \\
        Kaplan & 60 & 54 & 51 & 64 & 683 & 66 & 91 & 19 & 47 & 64 \\
        Lin & 403 & 403 & 403 & 400 & 291 & 403 & 403 & 403 & 403 & 400 \\
        Perry & 349 & 349 & 349 & 300 & 158 & 349 & 349 & 349 & 349 & 300 \\
        Ringuette & 21 & 46 & 45 & 13 & 291 & 22 & 327 & 30 & 17 & 13 \\
        Skeleton & 26 & 34 & 34 & 29 & 204 & 41 & 33 & 16 & 20 & 29 \\
        Staecker & 486 & 450 & 420 & 474 & 515 & 487 & 500 & 214 & 368 & 474 \\
        ZIC & 53 & 48 & 55 & 48 & 441 & 53 & 90 & 49 & 51 & 48 \\
        \bottomrule
    \end{tabular}
    \caption*{Lower RMS indicates more equitable profit distribution among traders.}
\end{table}

\begin{table}[htbp]
    \centering
    \caption{Self-Play RMSD (Root Mean Squared Deviation)}
    \label{table:s6_selfplay_rmsd}
    \begin{tabular}{lrrrrrrrrrr}
        \toprule
        Strategy & BASE & BBBS & BSSS & EQL & RAN & PER & SHRT & TOK & SML & LAD \\
        \midrule
        BGAN & 8 & 4 & 22 & 3 & 48 & 10 & 4 & 2 & 22 & 3 \\
        Breton & 40 & 40 & 40 & 36 & 392 & 40 & 40 & 40 & 40 & 36 \\
        EL & 12 & 11 & 13 & 12 & 60 & 26 & 10 & 2 & 11 & 12 \\
        Gamer & 25 & 25 & 25 & 21 & 503 & 25 & 25 & 25 & 25 & 21 \\
        Jacobson & 6 & 6 & 6 & 3 & 50 & 6 & 6 & 6 & 6 & 3 \\
        Kaplan & 49 & 51 & 49 & 51 & 565 & 51 & 60 & 19 & 46 & 51 \\
        Lin & 9 & 9 & 9 & 7 & 77 & 9 & 9 & 9 & 9 & 7 \\
        Perry & 11 & 11 & 11 & 8 & 30 & 11 & 11 & 11 & 11 & 8 \\
        Ringuette & 14 & 8 & 7 & 8 & 40 & 15 & 1 & 0 & 6 & 8 \\
        Skeleton & 21 & 26 & 26 & 22 & 160 & 30 & 24 & 16 & 18 & 22 \\
        Staecker & 9 & 9 & 10 & 7 & 270 & 9 & 10 & 1 & 9 & 7 \\
        ZIC & 35 & 30 & 33 & 32 & 349 & 35 & 34 & 11 & 27 & 32 \\
        \bottomrule
    \end{tabular}
    \caption*{Lower RMSD indicates better price convergence to equilibrium.}
\end{table}

\begin{table}[htbp]
    \centering
    \caption{Self-Play Trades per Period}
    \label{table:s6_selfplay_trades_per_period}
    \begin{tabular}{lrrrrrrrrrrr}
        \toprule
        Strategy & BASE & BBBS & BSSS & EQL & RAN & PER & SHRT & TOK & SML & LAD \\
        \midrule
        BGAN & 1.0 & 0.6 & 1.8 & 0.8 & 1.2 & 1.0 & 0.7 & 0.4 & 1.6 & 0.8 \\
        Breton & 2.3 & 2.3 & 2.3 & 1.9 & 1.8 & 2.3 & 2.3 & 2.3 & 2.3 & 1.9 \\
        EL & 1.7 & 0.8 & 1.9 & 1.5 & 1.8 & 1.1 & 0.7 & 0.1 & 1.5 & 1.5 \\
        Gamer & 1.1 & 1.1 & 1.1 & 1.0 & 1.6 & 1.1 & 1.1 & 1.1 & 1.1 & 1.0 \\
        Jacobson & 0.8 & 0.8 & 0.8 & 0.7 & 1.5 & 0.8 & 0.8 & 0.8 & 0.8 & 0.7 \\
        Kaplan & 2.0 & 1.0 & 3.0 & 1.9 & 1.8 & 2.2 & 1.2 & 0.5 & 1.8 & 1.9 \\
        Lin & 1.5 & 1.5 & 1.5 & 1.3 & 1.6 & 1.5 & 1.5 & 1.5 & 1.5 & 1.3 \\
        Perry & 1.4 & 1.4 & 1.4 & 1.3 & 1.7 & 1.4 & 1.4 & 1.4 & 1.4 & 1.3 \\
        Ringuette & 2.2 & 0.7 & 2.4 & 1.9 & 1.5 & 2.3 & 0.6 & 0.1 & 1.7 & 1.9 \\
        Skeleton & 2.1 & 1.0 & 3.1 & 1.9 & 1.9 & 2.3 & 2.2 & 0.5 & 1.9 & 1.9 \\
        Staecker & 0.9 & 0.4 & 1.5 & 0.8 & 1.4 & 0.9 & 0.7 & 0.1 & 1.1 & 0.8 \\
        ZIC & 1.7 & 0.7 & 2.5 & 1.5 & 1.8 & 1.9 & 1.1 & 0.3 & 1.4 & 1.5 \\
        \bottomrule
    \end{tabular}
    \caption*{Average number of trades completed per period in self-play.}
\end{table}

In contrast, fixed-rule agents such as Skeleton and Breton generally demonstrate highly stable and efficient self-play across environments. Their high trades per period, low RMSD, and minimal profit dispersion (Tables~\ref{table:s6_selfplay_trades_per_period}, \ref{table:s6_selfplay_rmsd}, and \ref{table:s6_selfplay_profit_dispersion}) signify their ability to foster orderly and equitable markets when interacting solely amongst themselves. This fundamental robustness highlights their role as reliable liquidity providers, a stark contrast to the destabilizing dynamics observed in homogeneous sniper populations. This supports the view that the competitive advantage of sniping strategies is parasitic, dependent on the presence of agents willing to provide liquidity within the market.

\subsection{Mixed Play Performance (Heterogeneous Round-Robin Tournament)}
Our analysis of the Heterogeneous Round-Robin Tournament, where all Santa Fe agents compete against each other, reveals insights into their overall ecological fitness. Table~\ref{table:s6_mixed_play_performance} summarizes the average rank and total wins for each strategy across the diverse market environments.

\begin{table}[htbp]
    \centering
    \caption{Mixed Play Overall Performance (Avg Rank and Wins)}
    \label{table:s6_mixed_play_performance}
    \begin{tabular}{lrr}
        \toprule
        Strategy & Average Rank & Total Wins \\
        \midrule
        ZIC & 6.64 & 54 \\
        Skeleton & 5.18 & 18 \\
        Kaplan & 6.99 & 10 \\
        Ringuette & 4.00 & 149 \\
        Gamer & 5.81 & 55 \\
        Perry & 5.00 & 56 \\
        Ledyard & 6.06 & 50 \\
        BGAN & 8.30 & 22 \\
        Staecker & 7.02 & 32 \\
        Jacobson & 7.20 & 14 \\
        Lin & 8.42 & 28 \\
        Breton & 7.39 & 12 \\
        \bottomrule
    \end{tabular}
    \caption*{Average rank and total wins for each strategy in the Heterogeneous Round-Robin Tournament across all environments.}
\end{table}

The original Santa Fe tournament notably highlighted the unexpected victory of Kaplan, a relatively simple heuristic, over more complex, AI-driven strategies. Our replication in a heterogeneous round-robin tournament is consistent with the competitive power of simple heuristics, though it reveals a notable shift in the most dominant sniper.

In our replication, Ringuette ranks highest in the overall tournament, achieving the best average rank (4.00) and the highest number of wins (149) across the ten diverse environments. This contrasts with Kaplan, which ranks 7th, suggesting that while sniping remains an effective strategy, Ringuette's specific implementation appears to confer a more robust competitive advantage in this contemporary setup. Further log analysis confirms these distinctions: while both agents employ opportunistic sniper tactics, Kaplan's behavior in homogeneous self-play often leads to prolonged deadlocks with extreme initial bids and late-period trades, a stark manifestation of the 'Sniper's Dilemma'. In contrast, Ringuette, while also susceptible to self-play inefficiencies in stressful environments (SHRT), demonstrates more consistent and less volatile price discovery in milder homogeneous settings and superior overall fitness in heterogeneous mixed-play, suggesting a more refined adaptability in its opportunistic decision-making. For instance, in Round 1 of the BASE environment, Ringuette demonstrated its opportunistic nature by executing a profitable trade with Staecker (price 144, buyer profit 148) and another with Lin (price 146, buyer profit 2), showcasing its ability to extract surplus even with small margins. Kaplan, too, exhibited this behavior, securing a notable profit in a trade with Breton (price 141, buyer profit 145). Other notably strong performers include Perry (2nd place) and Skeleton (3rd place), both representing relatively simple heuristic approaches. This suggests that the underlying principles behind the Kaplan Paradox may extend to a broader class of simple, robust, and opportunistic strategies, rather than being exclusive to Kaplan.





\subsection{Complexity and Performance Trade-offs}
Our results offer further insights into the relationship between a strategy's design complexity and its ecological fitness in competitive markets. The evidence suggests that increased strategic complexity does not consistently translate to superior performance, potentially incurring a "penalty for overhead."

While adaptive and belief-based models represent higher orders of cognitive sophistication, the overall Round-Robin tournament results tend to indicate that simpler, more direct heuristics frequently outperform them. The top three performers (Ringuette, Perry, Skeleton) embody relatively straightforward rule-sets. In contrast, cognitively more elaborate agents like Jacobson (equilibrium estimation, ranks 9th), Staecker (ranks 8th), BGAN (belief-based, ranks 11th), and Lin (statistical prediction, ranks 12th) generally occupy the lower half of the rankings. For example, in homogeneous self-play, Jacobson and BGAN frequently exhibit complete market breakdowns, failing to execute any trades across entire rounds, a clear indication that their sophisticated mechanisms can lead to inaction and inefficiency when interacting with similar complex counterparts.

This finding reinforces aspects of the "Kaplan Paradox" from a different perspective: the most ecologically fit strategies are often characterized by their simplicity, robustness, and speed of execution rather than their ability to model complex market dynamics or form sophisticated beliefs. The trade-off between the cognitive overhead of complex models and the ecological robustness of simpler heuristics appears to favor the latter in this competitive environment.



\subsection{Analysis of Evolutionary Dynamics}
Beyond static tournament performance, we investigate the long-term ecological dynamics of these strategies using an evolutionary tournament model. This experiment simulates multiple generations of competition, where successful strategies increase their population share and underperforming ones face elimination.

Our results from 10 evolutionary seeds reveal a clear hierarchy of evolutionary stability. Skeleton emerges as the most evolutionarily stable strategy, comprising 62.5\% of the final population across seeds. Its simple, robust fixed-rule approach consistently outperforms more complex or adaptive strategies in a dynamic competitive environment. Kaplan (13.4\%) and Ringuette (10\%) maintain stable presences in the final population. This indicates that their opportunistic, parasitic strategies are evolutionarily viable, albeit not dominant, suggesting they can coexist sustainably within the market ecosystem. ZIP, representing general adaptive learning, consistently goes extinct early (by generation 7.5 across all seeds). This suggests that despite its strong self-play and niche dominance in asymmetric static markets, its adaptive mechanism is not sufficiently robust or specialized to compete effectively in a dynamic evolutionary environment, leading to its elimination.

\begin{figure}[htbp]
    \centering
    \includegraphics[width=\textwidth]{figures/santafe_evolutionary_dynamics.pdf}
    \caption{Evolutionary dynamics of Santa Fe trading strategies over 50 generations, aggregated across 10 seeds. Lines show mean population share with shaded bands indicating one standard deviation. Skeleton (purple) dominates with 62.5\% of the final population. Snipers Kaplan (red) and Ringuette (orange) persist at lower levels. ZIP (green dotted) goes extinct early, typically by generation 8. The emergence of Skeleton dominance and ZIP extinction demonstrates that simple, robust fixed-rule strategies outcompete both opportunistic snipers and general adaptive learners in an evolutionary competition.}
    \label{fig:s6_evolutionary_dynamics}
\end{figure}

\subsection{Summary and Conclusion}
Our replication of the Santa Fe Tournament, analyzed through the lens of foundational literature, offers the following key insights into the ecological dynamics of heuristic-based trading strategies:

\begin{enumerate}
    \item Simple, rule-based strategies, particularly those employing effective opportunistic behavior (e.g., Perry, Gamer, Ringuette), demonstrate substantial exploitative power and competitive advantage in heterogeneous market ecosystems. Their success is associated with effective opportunistic behavior. While Kaplan is historically noted for sniper tactics, our easy play results suggest other agents achieve higher exploitative ratios against ZIC baselines.
    \item The collective behavior of opportunistic or complex strategies, not limited to purely sniper populations, can lead to severe market inefficiency and liquidity collapse in homogeneous environments. Our self-play experiments reveal that many adaptive and belief-based agents (e.g., Jacobson, Staecker, BGAN, Gamer, Perry, Lin, EL) exhibit significant market breakdown, characterized by low efficiency, negligible trades, and high volatility, even in the BASE environment. This confirms that the success of such strategies is fundamentally dependent on the presence of diverse or more liquidity-providing agents. In stark contrast, simple fixed-rule agents like Skeleton and Breton consistently foster highly stable and efficient self-play markets, underscoring their inherent robustness and role as reliable liquidity providers.
    \item There is no clear evidence that increased strategic complexity consistently leads to superior ecological fitness. Instead, simpler, robust heuristics like Ringuette, Perry, and Skeleton often prove highly effective, suggesting that an optimal balance of simplicity and exploitative capability is favored in this competitive environment. The marked struggles of more cognitively elaborate agents (e.g., Jacobson, Staecker, BGAN, Lin, EL, Gamer, Perry) in homogeneous self-play, characterized by significant drops in efficiency and liquidity, further underscore a "penalty for overhead" associated with such complexity when robust self-organization is required.
    \item Our analysis of general adaptive strategies (ZIP) in an extended round-robin tournament suggests that while not universally dominant, ZIP excels in asymmetric market structures. However, in evolutionary competition, ZIP consistently faces early extinction, suggesting that its adaptive mechanisms are outcompeted by more specialized and robust heuristics over multiple generations.
\end{enumerate}
