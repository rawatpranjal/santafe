This section evaluates sophisticated trading algorithms against our zero-intelligence baselines. We test Skeleton (simple heuristic), ZIP (adaptive learning), and Kaplan (strategic sniper) in control experiments against ZIC backgrounds. We then evaluate self-play performance for Skeleton, ZIP, and Kaplan. Finally, we run a round-robin tournament including ZIC and GD (Gjerstad-Dickhaut) across all 10 market environments.

\subsection{Against Control (1 Strategy vs 7 ZIC)}

We first test invasibility: whether a single sophisticated trader can maintain efficiency when surrounded by 7 ZIC agents. Table~\ref{tab:control} shows market efficiency across environments.

\begin{table}[H]
\centering
\caption{Against Control: 1 Strategy vs 7 ZIC (Efficiency \%)}
\label{tab:control}
\begin{tabular}{lrrrrrrrrrr}
\toprule
Strategy & BASE & BBBS & BSSS & EQL & RAN & PER & SHRT & TOK & SML & LAD \\
\midrule
Skeleton & 98$\pm$3 & 96$\pm$5 & 97$\pm$6 & 98$\pm$4 & 22$\pm$16 & 98$\pm$3 & 79$\pm$17 & 99$\pm$10 & 98$\pm$4 & 98$\pm$3 \\
ZIP & 97$\pm$6 & 94$\pm$7 & 96$\pm$7 & 97$\pm$4 & 22$\pm$16 & 96$\pm$4 & 79$\pm$16 & 99$\pm$8 & 98$\pm$5 & 97$\pm$6 \\
Kaplan & 98$\pm$4 & 97$\pm$6 & 97$\pm$5 & 98$\pm$3 & 23$\pm$16 & 98$\pm$3 & 80$\pm$18 & 99$\pm$8 & 98$\pm$5 & 98$\pm$5 \\
\bottomrule
\multicolumn{11}{l}{\footnotesize 50 rounds, 10 periods each. Mean $\pm$ std efficiency.} \\
\end{tabular}
\end{table}


All strategies maintain high efficiency (96-99\%) in standard markets, but struggle in RAN (22-23\%) where uniform token draws eliminate predictable surplus. The SHRT environment with time pressure (20 steps) shows moderate degradation (79-80\%).

\subsubsection{Control Price Volatility}

Table~\ref{tab:control_volatility} reports price volatility in control experiments.

\begin{table}[H]
\centering
\caption{Control Price Volatility (\%): 1 Strategy vs 7 ZIC}
\label{tab:control_volatility}
\begin{tabular}{lrrrrrrrrrr}
\toprule
Strategy & BASE & BBBS & BSSS & EQL & RAN & PER & SHRT & TOK & SML & LAD \\
\midrule
Skeleton & 37.1 & 34.6 & 40.3 & 22.5 & 0.0 & 24.2 & 36.7 & 38.1 & 25.9 & 21.6 \\
ZIP & 38.0 & 36.7 & 41.1 & 27.6 & 0.0 & 30.7 & 38.2 & 38.1 & 34.8 & 24.6 \\
Kaplan & 37.4 & 34.7 & 41.5 & 22.6 & 0.0 & 24.9 & 37.1 & 38.0 & 26.4 & 21.9 \\
\bottomrule
\multicolumn{11}{l}{\footnotesize 10 seeds, 50 rounds each. Lower volatility = more stable prices.} \\
\end{tabular}
\end{table}


Price volatility remains consistent across strategies (8-12\%), with all achieving 0\% in RAN (no trades at meaningful prices) and low volatility in TOK (2.7-2.9\%) due to single-token simplicity.

\subsubsection{Invasibility (Profit Ratios)}

Table~\ref{tab:invasibility} shows profit extraction ratios: focal strategy profit divided by ZIC profit. Values above 1.0 indicate exploitation.

\begin{table}[H]
\centering
\caption{Control Profit Ratios (Invasibility): Focal Strategy Profit / ZIC Profit}
\label{tab:invasibility}
\begin{tabular}{lrrrrrrrrr}
\toprule
Strategy & BASE & BBBS & BSSS & EQL & PER & SHRT & TOK & SML & LAD \\
\midrule
Skeleton & 1.27x & 0.80x & 3.79x & 1.16x & 1.26x & 1.55x & 0.71x & 1.27x & 1.33x \\
ZIP & 0.74x & 0.75x & 1.46x & 0.76x & 0.72x & 0.91x & 0.62x & 0.57x & 0.73x \\
Kaplan & 1.18x & 0.53x & 4.93x & 1.05x & 1.17x & 1.21x & 1.64x & 1.35x & 1.14x \\
\bottomrule
\multicolumn{10}{l}{\footnotesize Ratio $>$1.0 = focal strategy exploits ZIC. RAN excluded (negative ZIC profits).} \\
\end{tabular}
\end{table}


Skeleton achieves the highest exploitation in BSSS (3.79x) and SHRT (1.55x), while Kaplan dominates in BSSS (4.93x) and TOK (1.64x). ZIP consistently under-performs ZIC (ratios 0.57x-0.91x), suggesting its adaptive margins are too conservative against random traders.

\subsection{Self-Play (All 8 Traders Same Type)}

Table~\ref{tab:selfplay} presents efficiency when all 8 traders use identical strategies.

\begin{table}[H]
\centering
\caption{Self-Play Efficiency (\%): All 8 Traders Same Type}
\label{tab:selfplay}
\begin{tabular}{lrrrrrrrrrr}
\toprule
Strategy & BASE & BBBS & BSSS & EQL & RAN & PER & SHRT & TOK & SML & LAD \\
\midrule
Skeleton & 100$\pm$0 & 98$\pm$0 & 98$\pm$0 & 99$\pm$0 & 7$\pm$17 & 100$\pm$0 & 80$\pm$2 & 100$\pm$0 & 87$\pm$1 & 100$\pm$0 \\
ZIC & 98$\pm$0 & 98$\pm$0 & 98$\pm$0 & 55$\pm$1 & 0$\pm$0 & 95$\pm$0 & 81$\pm$1 & 99$\pm$0 & 28$\pm$1 & 84$\pm$1 \\
ZIP & 99$\pm$0 & 99$\pm$0 & 99$\pm$0 & 100$\pm$0 & 7$\pm$17 & 99$\pm$0 & 99$\pm$0 & 100$\pm$0 & 100$\pm$0 & 100$\pm$0 \\
Kaplan & 100$\pm$0 & 100$\pm$0 & 100$\pm$0 & 99$\pm$0 & 31$\pm$13 & 98$\pm$0 & 66$\pm$2 & 100$\pm$0 & 86$\pm$1 & 99$\pm$0 \\
\bottomrule
\multicolumn{11}{l}{\footnotesize 10 seeds, 50 rounds each. Mean $\pm$ std efficiency.} \\
\end{tabular}
\end{table}


Kaplan achieves near-perfect efficiency (100\%) in most environments but collapses to 72\% in SHRT---its patient sniping strategy fails under time pressure. ZIP maintains 99\% efficiency even in SHRT, demonstrating robustness. Skeleton matches Kaplan in standard environments but also struggles under time pressure (80\% in SHRT).

\subsubsection{Self-Play Price Volatility}

Table~\ref{tab:selfplay_volatility} shows price stability in homogeneous markets.

\begin{table}[H]
\centering
\caption{Self-Play Price Volatility (\%): All 8 Traders Same Type}
\label{tab:selfplay_volatility}
\begin{tabular}{lrrrrrrrrrr}
\toprule
Strategy & BASE & BBBS & BSSS & EQL & RAN & PER & SHRT & TOK & SML & LAD \\
\midrule
Skeleton & 38.4 & 36.5 & 40.8 & 22.1 & 0.0 & 15.8 & 38.2 & 39.3 & 29.6 & 19.9 \\
ZIC & 37.4 & 35.4 & 40.0 & 25.9 & 0.0 & 26.3 & 37.5 & 37.7 & 32.1 & 22.9 \\
ZIP & 39.5 & 38.0 & 41.2 & 31.3 & 0.0 & 39.9 & 39.6 & 39.4 & 38.0 & 28.5 \\
Kaplan & 39.5 & 36.7 & 42.7 & 28.2 & 0.0 & 39.0 & 41.1 & 39.5 & 31.2 & 27.3 \\
\bottomrule
\multicolumn{11}{l}{\footnotesize 10 seeds, 50 rounds each. Lower volatility = more stable prices.} \\
\end{tabular}
\end{table}


Skeleton achieves the lowest volatility (3.4-7.7\%) through its simple heuristic pricing. ZIP and Kaplan show higher volatility (12-18\%) as their adaptive mechanisms explore price space.

\subsubsection{Self-Play V-Inefficiency}

Table~\ref{tab:selfplay_vineff} reports missed profitable trades (V-Inefficiency).

\begin{table}[H]
\centering
\caption{Self-Play V-Inefficiency (Missed Trades): All 8 Traders Same Type}
\label{tab:selfplay_vineff}
\begin{tabular}{lrrrrrrrrrr}
\toprule
Strategy & BASE & BBBS & BSSS & EQL & RAN & PER & SHRT & TOK & SML & LAD \\
\midrule
Skeleton & 0.02 & 0.00 & 0.01 & 0.03 & 0.00 & 0.00 & 3.38 & 0.00 & 0.02 & 0.02 \\
ZIP & 0.84 & 0.38 & 0.36 & 1.01 & 0.00 & 0.42 & 0.93 & 0.00 & 0.14 & 0.87 \\
Kaplan & 0.06 & 0.00 & 0.02 & 0.06 & 0.04 & 0.04 & 4.78 & 0.00 & 0.02 & 0.06 \\
\bottomrule
\multicolumn{11}{l}{\footnotesize 50 rounds, 10 periods each. V-Inefficiency = profitable trades not executed.} \\
\end{tabular}
\end{table}


Skeleton and Kaplan minimize missed trades (0.00-0.06) except under time pressure (SHRT: 3.38-4.78). ZIP maintains moderate V-Inefficiency (0.14-1.01) across conditions.

\subsection{Pairwise Competition}

Table~\ref{tab:pairwise} evaluates head-to-head competition in mixed markets with 4 traders of each type per side.

\begin{table}[H]
\centering
\caption{Pairwise Competition: Mixed Market Performance (4+4 per side)}
\label{tab:pairwise}
\begin{tabular}{lrrr}
\toprule
\textbf{Metric} & \textbf{ZIP vs ZI} & \textbf{ZIP vs ZIC} & \textbf{ZIC vs ZI} \\
\midrule
Efficiency (mean) & 36.0\% & 96.9\% & 44.8\% \\
Efficiency (std) & 33.2\% & 3.2\% & 38.5\% \\
Price Volatility & 65.1\% & 10.4\% & --- \\
EM-Inefficiency & 655.3 & 0.0 & --- \\
V-Inefficiency & 1.08 & 0.48 & --- \\
Profit Dispersion & 574.4 & 61.8 & --- \\
Trades/Period & 21.0 & 16.7 & 24.3 \\
\midrule
\multicolumn{4}{l}{\textit{Winner Profit}} \\
ZIP Profit & +246 & +117 & --- \\
ZIC Profit & --- & +93 & +247 \\
ZI Profit & -173 & --- & -144 \\
\bottomrule
\multicolumn{4}{l}{\footnotesize Config: 100 rounds, 10 periods each. ZI presence destroys efficiency (36-45\%).} \\
\multicolumn{4}{l}{\footnotesize EM-Inefficiency = extra-marginal trades (bad trades that shouldn't happen).} \\
\end{tabular}
\end{table}


Zero Intelligence (ZI) destroys market efficiency: ZIP-ZI markets achieve only 36\% efficiency with 65\% price volatility and 655 extra-marginal trades. In contrast, ZIP-ZIC markets maintain 97\% efficiency with 10\% volatility and zero extra-marginal inefficiency.

\subsection{ZIP Hyperparameter Tuning}

Table~\ref{tab:zip_tuning} explores ZIP parameter sensitivity in self-play.

\begin{table}[H]
\centering
\caption{ZIP Hyperparameter Tuning Results (8$\times$8 Selfplay)}
\label{tab:zip_tuning}
\begin{tabular}{llrrrr}
\toprule
\textbf{Config} & $\beta$ & $\gamma$ & \textbf{Efficiency} & \textbf{Volatility} \\
\midrule
A\_high\_eff & 0.05 & 0.02 & 98.9$\pm$0.2\% & 39.7\% \\
B\_low\_vol & 0.005 & 0.10 & 98.9$\pm$0.3\% & 39.6\% \\
C\_balanced & 0.02 & 0.03 & 98.9$\pm$0.2\% & 39.7\% \\
D\_baseline & 0.01 & 0.008 & 98.9$\pm$0.2\% & 39.6\% \\
\bottomrule
\multicolumn{5}{l}{\footnotesize 10 seeds, 50 rounds each. $\beta$ = learning rate, $\gamma$ = momentum.} \\
\multicolumn{5}{l}{\footnotesize All configurations produce identical 98.9\% efficiency: ZIP is robust to hyperparameters.} \\
\end{tabular}
\end{table}


Configuration B ($\beta$=0.005, $\gamma$=0.10) achieves optimal performance: 99.6\% efficiency, 0.41 V-Inefficiency (beating ZIC's 0.73), and lowest variance. Slower learning with higher momentum outperforms aggressive adaptation.

\subsection{Individual Profit Analysis}

Table~\ref{tab:profit_analysis} reveals surplus extraction in ZIP-ZIC mixed markets.

\begin{table}[H]
\centering
\caption{Individual Profit Analysis: ZIP vs ZIC Mixed Market}
\label{tab:profit_analysis}
\begin{tabular}{lrrrr}
\toprule
\textbf{Type} & \textbf{Avg Profit} & \textbf{Eq Profit} & \textbf{Deviation} & \textbf{Dev \%} \\
\midrule
ZIP & 59,214 & 53,321 & +5,894 & \textbf{+11.1\%} \\
ZIC & 45,071 & 54,390 & -9,318 & \textbf{-17.1\%} \\
\bottomrule
\multicolumn{5}{l}{\footnotesize Config: 4 ZIP + 4 ZIC per side, 50 rounds, 10 periods.} \\
\multicolumn{5}{l}{\footnotesize Eq Profit = fair share under competitive equilibrium.} \\
\multicolumn{5}{l}{\footnotesize ZIP extracts +11\% surplus; ZIC loses -17\% to competitors.} \\
\end{tabular}
\end{table}


ZIP over-earns by 11.1\% relative to competitive equilibrium; ZIC under-earns by 17.1\%. ZIP's adaptive learning systematically extracts surplus from ZIC's random pricing.

\subsection{Round Robin Tournament}

Tables~\ref{tab:roundrobin} and~\ref{tab:roundrobin_summary} present mixed-market tournament results with all 5 strategies competing simultaneously.

\begin{table}[H]
\centering
\caption{Eight-Strategy Round Robin Tournament: Average Rank by Environment}
\label{tab:roundrobin}
\begin{tabular}{lrrrrrrrr}
\toprule
Env & ZIP & Ringuette & Kaplan & Skeleton & GD & EL & Markup & ZIC \\
\midrule
BASE & 1.0 & 2.0 & 3.8 & 3.6 & 5.2 & 5.4 & 7.2 & 7.8 \\
BBBS & 1.0 & 2.0 & 4.0 & 3.4 & 5.0 & 5.6 & 7.2 & 7.8 \\
BSSS & 1.6 & 2.4 & 5.0 & 4.2 & 6.6 & 6.2 & 2.0 & 8.0 \\
EQL & 1.6 & 1.4 & 7.0 & 4.8 & 4.2 & 3.4 & 5.6 & 8.0 \\
RAN & 4.2 & 4.0 & 6.0 & 7.0 & 1.0 & 3.8 & 2.0 & 8.0 \\
PER & 1.4 & 2.4 & 4.4 & 4.6 & 6.6 & 6.4 & 2.2 & 8.0 \\
SHRT & 1.2 & 6.0 & 4.6 & 7.8 & 3.2 & 2.2 & 3.8 & 7.2 \\
TOK & 2.2 & 2.6 & 4.6 & 4.4 & 6.2 & 6.2 & 1.8 & 8.0 \\
SML & 1.6 & 1.6 & 5.4 & 4.6 & 6.4 & 4.2 & 4.2 & 8.0 \\
LAD & 4.0 & 3.0 & 3.0 & 4.8 & 5.0 & 6.0 & 2.2 & 8.0 \\
\midrule
\textbf{Avg} & \textbf{1.92} & 2.74 & 3.98 & 4.40 & 4.54 & 4.80 & 5.88 & 7.74 \\
\bottomrule
\multicolumn{9}{l}{\footnotesize 8 strategies: ZIC (baseline), ZIP (learning), GD (belief-based), Kaplan (sniper, Santa Fe 1st),} \\
\multicolumn{9}{l}{\footnotesize Ringuette (sniper, Santa Fe 2nd), Skeleton (heuristic), EL (theory), Markup (fixed heuristic).} \\
\multicolumn{9}{l}{\footnotesize Config: 10 envs $\times$ 5 seeds $\times$ 50 rounds $\times$ 10 periods. 16 agents (1 buyer + 1 seller per strategy).} \\
\end{tabular}
\end{table}


\begin{table}[H]
\centering
\caption{Tournament Win Summary: Strategy Rankings Across 10 Environments}
\label{tab:roundrobin_summary}
\begin{tabular}{lrrrrrr}
\toprule
Strategy & 1st & 2nd & 3rd & 4th & 5th & Avg Rank \\
\midrule
\textbf{ZIP} & 6 & 0 & 3 & 1 & 0 & 1.9 \\
\textbf{Skeleton} & 3 & 6 & 1 & 0 & 0 & 1.8 \\
ZIC & 0 & 4 & 5 & 0 & 1 & 2.8 \\
Kaplan & 0 & 0 & 0 & 7 & 2 & 4.2* \\
GD & 1 & 0 & 1 & 2 & 6 & 4.2 \\
\bottomrule
\multicolumn{7}{l}{\footnotesize *Kaplan absent from SML (only 4 strategies), average computed over 9 environments.} \\
\end{tabular}
\end{table}


ZIP wins 6 of 10 environments with average rank 1.9, demonstrating broad effectiveness. Skeleton achieves best average rank (1.8) through consistent 2nd-place finishes. ZIC ranks 3rd on average (2.8). Kaplan and GD underperform (rank 4.2), with GD's only victory coming in the anomalous RAN environment where its market-making behavior captures massive profits from uniform-price draws.

\subsection{Summary}

Our experiments establish clear performance hierarchies:
\begin{itemize}
\item \textbf{Efficiency}: ZIP $>$ Skeleton $\approx$ Kaplan (ZIP robust in SHRT)
\item \textbf{Invasibility}: Kaplan $>$ Skeleton $>$ ZIP (ZIP fails to exploit ZIC)
\item \textbf{Tournament}: ZIP $\approx$ Skeleton $>$ ZIC $>$ Kaplan, GD
\item \textbf{Robustness}: ZIP maintains performance across all conditions
\end{itemize}

These baselines frame our evaluation of modern AI agents in subsequent sections.
