To test foundational hypotheses of market behavior, we replicate and extend the classic experiments of Gode and Sunder (1993) and Cliff and Bruten (1997). We deploy a hierarchy of five zero-intelligence (ZI) strategies into the ten diverse market environments of the Santa Fe tournament. This allows us to systematically evaluate the marginal value of specific cognitive abilities, from simple budget-adherence to adaptive learning.

The hierarchy of strategies is as follows:
\begin{itemize}
    \item \textbf{ZI (Unconstrained):} Submits bids and asks uniformly at random from the full price range [1, 1000]. This agent acts as a control, representing pure noise trading without economic rationality.
    \item \textbf{ZIC (Budget-Constrained):} Submits random prices like ZI but is constrained to never make an unprofitable trade. This replicates the core "Zero-Intelligence Constrained" agent from Gode and Sunder (1993).
    \item \textbf{ZIC2 (Market-Aware):} A ZIC agent that also observes the order book and only submits a random price if it improves the current best bid or ask.
    \item \textbf{ZIP (Adaptive):} A ZIC-style agent that incorporates a simple learning rule (Widrow-Hoff) to adapt its profit margin based on trading success, as introduced by Cliff and Bruten (1997).
    \item \textbf{ZIP2 (Adaptive + Market-Aware):} Combines ZIP's adaptive learning with ZIC2's market awareness.
\end{itemize}

\subsection{Revisiting the Gode-Sunder Hypothesis: The Role of Budget Constraints}
First, we test the Gode and Sunder (1993) hypothesis that allocative efficiency is primarily a feature of the market institution, not the intelligence of its participants. The hypothesis posits that forbidding agents from making unprofitable trades is sufficient to achieve high levels of market efficiency.

\subsubsection{Performance in Non-Strategic Environments}
We begin by testing the strategies against passive "TruthTeller" sellers who offer tokens at their true cost. This "easy-play" setting isolates the buyers' search problem against non-strategic opponents.

The results, presented in Table~\ref{tab:easyplay_full}, are consistent with the Gode-Sunder hypothesis. The unconstrained ZI agent exhibits low allocative efficiency, achieving only 29\% in the BASE environment. Transaction logs (Figure~\ref{fig:easyplay_dynamics}, panel 1) show that ZI's high trade volume (16 trades per period vs. 8 profitable opportunities) is driven by deeply unprofitable, extra-marginal trades that destroy surplus.

The addition of a single budget constraint significantly improves market outcomes. The ZIC agent achieves 93--99\% efficiency across all ten environments. As Gode and Sunder found, the institutional rule—not trader savvy—is a principal driver of allocative efficiency. Further additions of intelligence (ZIC2, ZIP) provide only marginal gains in this setting, pushing efficiency to nearly 100\%.

\begin{table}[h]
\centering
\caption{Easy-play allocative efficiency (\%) across all market environments. Buyers compete against passive TruthTeller sellers who ask at true cost. ZI is wildly inefficient (11--67\%) because it accepts loss-making trades. Adding a budget constraint (ZIC1) achieves 93--99\% efficiency. Market awareness (ZIC2) and learning (ZIP1, ZIP2) saturate efficiency at 100\% in all environments.}
\label{tab:easyplay_full}
\small
\begin{tabular}{lcccccccccc}
\toprule
Strategy & BASE & BBBS & BSSS & EQL & RAN & PER & SHRT & TOK & SML & LAD \\
\midrule
ZI   & 29  & 52  & 67  & 25  & 11  & 29  & 29  & 95  & 27  & 25 \\
ZIC1 & 99  & 96  & 99  & 97  & 99  & 98  & 94  & 93  & 98  & 97 \\
ZIC2 & 100 & 99  & 100 & 100 & 100 & 100 & 100 & 100 & 100 & 100 \\
ZIP1 & 100 & 100 & 100 & 100 & 100 & 100 & 100 & 100 & 100 & 100 \\
ZIP2 & 100 & 100 & 100 & 100 & 100 & 100 & 100 & 100 & 100 & 100 \\
\bottomrule
\end{tabular}
\end{table}


\begin{figure}[h]
\centering
\includegraphics[width=\textwidth]{figures/easyplay_dynamics.pdf}
\caption{Easy-play market dynamics for five zero-intelligence strategies against passive TruthTeller sellers. ZI produces scattered prices across the full range with many loss-making trades; ZIC constrains trading to profitable prices only; ZIP achieves immediate execution through learning.}
\label{fig:easyplay_dynamics}
\end{figure}

\subsubsection{Performance under Strategic Pressure}
In a more challenging self-play setting, where all traders employ the same strategy, the budget constraint remains the most critical factor for efficiency. Table~\ref{tab:efficiency_full} shows ZIC's efficiency reaches 91\% in the BASE environment, a 64-point increase from ZI's 27\%. This demonstrates the robustness of the Gode-Sunder finding. Again, ZI's high trading volume (Table~\ref{tab:trades_full}) is shown to be counter-productive, consisting of surplus-destroying transactions, as visualized by the scattered prices in Figure~\ref{fig:selfplay_dynamics}. The results suggest that the most significant leap in market performance comes from the mechanical imposition of budget constraints.

\begin{table}[H]
\centering
\caption{Selfplay Allocative Efficiency (\%) Across All Market Environments}
\label{tab:efficiency_full}
\begin{tabular}{lccccc}
\toprule
\textbf{Environment} & \textbf{ZI} & \textbf{ZIC1} & \textbf{ZIC2} & \textbf{ZIP1} & \textbf{ZIP2} \\
\midrule
BASE & 27$\pm$2 & 91$\pm$2 & 95$\pm$1 & 100$\pm$0 & 100$\pm$0 \\
BBBS & 53$\pm$2 & 83$\pm$2 & 88$\pm$2 & 100$\pm$0 & 100$\pm$0 \\
BSSS & 53$\pm$2 & 88$\pm$1 & 92$\pm$1 & 100$\pm$0 & 100$\pm$0 \\
EQL & 29$\pm$4 & 92$\pm$1 & 95$\pm$1 & 100$\pm$0 & 100$\pm$2 \\
RAN & 13$\pm$1 & 99$\pm$0 & 99$\pm$0 & 100$\pm$0 & 100$\pm$0 \\
PER & 27$\pm$2 & 91$\pm$2 & 94$\pm$2 & 100$\pm$0 & 100$\pm$0 \\
SHRT & 27$\pm$2 & 66$\pm$2 & 76$\pm$2 & 100$\pm$0 & 100$\pm$1 \\
TOK & 94$\pm$2 & 75$\pm$3 & 81$\pm$3 & 100$\pm$0 & 100$\pm$0 \\
SML & 29$\pm$3 & 87$\pm$1 & 91$\pm$1 & 100$\pm$0 & 100$\pm$0 \\
LAD & 29$\pm$4 & 92$\pm$1 & 95$\pm$1 & 100$\pm$0 & 100$\pm$2 \\
\bottomrule
\multicolumn{6}{l}{\footnotesize Mean $\pm$ std over 10 seeds $\times$ 100 rounds. ZIP1/ZIP2 achieve 100\% everywhere.} \\
\end{tabular}
\end{table}

\begin{table}[H]
\centering
\caption{Trades per Period Across All Environments}
\label{tab:trades_full}
\begin{tabular}{lrrr}
\toprule
\textbf{Environment} & \textbf{ZI} & \textbf{ZIC} & \textbf{ZIP} \\
\midrule
BASE & 16.0 & 7.9 & 7.5 \\
BBBS & 8.0 & 5.9 & 5.6 \\
BSSS & 8.0 & 5.7 & 5.5 \\
EQL & 16.0 & 0.4 & 16.0 \\
RAN & 16.0 & 11.7 & 9.9 \\
PER & 16.0 & 7.9 & 7.8 \\
SHRT & 15.9 & 5.3 & 7.5 \\
TOK & 4.0 & 2.0 & 2.1 \\
SML & 8.0 & 3.4 & 3.0 \\
LAD & 16.0 & 7.9 & 7.5 \\
\bottomrule
\multicolumn{4}{l}{\footnotesize ZI trades all tokens; ZIC/ZIP are selective.} \\
\end{tabular}
\end{table}


\begin{figure}[h]
\centering
\includegraphics[width=\textwidth]{figures/selfplay_dynamics.pdf}
\caption{Selfplay market dynamics for five zero-intelligence strategies. ZI produces scattered prices across the full range; ZIC constrains trading near equilibrium; ZIP achieves tight convergence through learning.}
\label{fig:selfplay_dynamics}
\end{figure}

\subsection{Revisiting the Cliff-Bruten Hypothesis: The Role of Adaptive Intelligence}
While the budget constraint is sufficient for high allocative efficiency, Cliff and Bruten (1997) argued that intelligence, in the form of adaptive learning, is essential for achieving the price stability and coherence observed in human markets. Our results support a refined version of this hypothesis: intelligence is critical for the *speed and certainty* of convergence, thereby reducing coordination failures.

\subsubsection{Coordination Failures in Non-Adaptive Markets}
While ZIC is highly efficient, its random-search mechanism is slow and prone to failure under scarcity. This weakness is most apparent in the \textbf{SHRT} environment, where its efficiency drops to 66\% (Table~\ref{tab:efficiency_full}). The V-Inefficiency metric (Table~\ref{tab:vineff_full}), which counts profitable but unexecuted trades, reveals the mechanism: in SHRT, ZIC misses 227 intra-marginal trades per period. This pattern persists in other scarce environments like TOK (42 missed trades) and SML (37 missed trades). Because ZIC agents rely on chance for a bid and ask to cross, their ability to coordinate degrades when opportunities are limited by time or market size.

\subsubsection{Adaptive Learning and Market Coherence}
Adaptive learning (ZIP) largely resolves this coordination problem. The ZIP strategy achieves nearly 100\% allocative efficiency across all ten environments, with V-Inefficiency scores close to zero universally (Table~\ref{tab:vineff_full}). This is also reflected in the "easy-play" experiments (Table~\ref{tab:easyplay_time}), where ZIP's mean trade time is just 1.0 step, while ZIC's is 1.6-3.1 steps. ZIP does not need to search for the price; it learns it, leading to faster and more certain trade execution.

\subsubsection{Behavioral Signatures and Price Dynamics}
The behavioral signatures in Table~\ref{tab:selfplay_behavior} show that ZIP is strongly JUMP-dominant (57\%), actively trying to improve the price. Its Price Improvement Rate (PIR) of just 1\% indicates it has learned to avoid submitting bids that will not cross the spread. In a departure from the original Cliff and Bruten (1997) findings, our results show that this superior coordination does not necessarily lead to lower price volatility. In the BASE environment, ZIP's price volatility is 12\%, compared to ZIC's 7\% (Table~\ref{tab:volatility_full}). The primary contribution of adaptive intelligence in our experiments is the near-elimination of coordination failures, ensuring the market clears quickly and completely.

\begin{table}[H]
\centering
\caption{V-Inefficiency (Missed Trades) Across All Environments}
\label{tab:vineff_full}
\begin{tabular}{lrrr}
\toprule
\textbf{Environment} & \textbf{ZI} & \textbf{ZIC} & \textbf{ZIP} \\
\midrule
BASE & 0.0 & 0.3 & 0.5 \\
BBBS & 0.0 & 0.2 & 0.4 \\
BSSS & 0.0 & 0.2 & 0.3 \\
EQL & 0.0 & 0.0 & 0.0 \\
RAN & 0.0 & 0.0 & 1.6 \\
PER & 0.0 & 0.2 & 0.2 \\
SHRT & 0.0 & 2.7 & 0.6 \\
TOK & 0.0 & 0.1 & 0.0 \\
SML & 0.0 & 0.6 & 1.0 \\
LAD & 0.0 & 0.3 & 0.5 \\
\bottomrule
\multicolumn{4}{l}{\footnotesize Missed intra-marginal trades per period.} \\
\end{tabular}
\end{table}

\begin{table}[H]
\centering
\caption{Price Volatility (\%) Across All Market Environments}
\label{tab:volatility_full}
\begin{tabular}{lrrr}
\toprule
\textbf{Environment} & \textbf{ZI} & \textbf{ZIC} & \textbf{ZIP} \\
\midrule
BASE & 64$\pm$1 & 8$\pm$0 & 12$\pm$1 \\
BBBS & 51$\pm$1 & 7$\pm$0 & 11$\pm$0 \\
BSSS & 79$\pm$1 & 8$\pm$1 & 12$\pm$1 \\
EQL & 64$\pm$1 & 0$\pm$0 & 0$\pm$0 \\
RAN & 64$\pm$1 & 34$\pm$1 & 53$\pm$1 \\
PER & 65$\pm$2 & 8$\pm$1 & 13$\pm$1 \\
SHRT & 65$\pm$0 & 8$\pm$0 & 12$\pm$1 \\
TOK & 56$\pm$1 & 2$\pm$0 & 4$\pm$1 \\
SML & 57$\pm$0 & 23$\pm$2 & 36$\pm$3 \\
LAD & 64$\pm$1 & 8$\pm$0 & 12$\pm$1 \\
\bottomrule
\multicolumn{4}{l}{\footnotesize Volatility = price std / mean. Lower is better.} \\
\end{tabular}
\end{table}

\begin{table}[h]
\centering
\caption{Easy-play mean trade time (steps) across all market environments. Lower values indicate faster search. ZI and ZIP1/ZIP2 execute immediately (1.0 steps) but for opposite reasons: ZI accepts any price indiscriminately, while ZIP learns to target the seller's ask precisely. ZIC1 requires 1.6--3.1 steps because random sampling within budget bounds is a geometric waiting time problem.}
\label{tab:easyplay_time}
\small
\begin{tabular}{lcccccccccc}
\toprule
Strategy & BASE & BBBS & BSSS & EQL & RAN & PER & SHRT & TOK & SML & LAD \\
\midrule
ZI   & 1.0 & 1.0 & 1.1 & 1.0 & 1.0 & 1.0 & 1.0 & 1.0 & 1.1 & 1.0 \\
ZIC1 & 1.6 & 1.4 & 2.6 & 2.3 & 1.0 & 1.6 & 1.7 & 3.1 & 2.4 & 1.9 \\
ZIC2 & 1.2 & 1.2 & 1.4 & 1.2 & 1.0 & 1.5 & 1.2 & 1.3 & 1.3 & 1.2 \\
ZIP1 & 1.0 & 1.0 & 1.0 & 1.0 & 1.0 & 1.0 & 1.0 & 1.0 & 1.0 & 1.0 \\
ZIP2 & 1.0 & 1.0 & 1.0 & 1.0 & 1.0 & 1.0 & 1.0 & 1.0 & 1.0 & 1.0 \\
\bottomrule
\end{tabular}
\end{table}

\begin{table}[H]
\centering
\caption{Selfplay Behavioral Signatures (All 8 agents same strategy, 5 seeds $\times$ 5 periods)}
\label{tab:selfplay_behavior}
\begin{tabular}{lccccccr}
\toprule
\textbf{Strategy} & \textbf{Dominant Action} & \textbf{Trade Time} & \textbf{Early\%} & \textbf{PASS\%} & \textbf{SR} & \textbf{PIR} & \textbf{Profit/Trade} \\
\midrule
ZI & PASS (88\%) & 8.0 & 100 & 88 & $-$0.41 & 35 & $-$92.3 \\
ZIC1 & JUMP (45\%) & 23.8 & 75 & 5 & $-$0.12 & 7 & 51.2 \\
ZIC2 & JUMP (46\%) & 13.4 & 90 & 16 & 0.08 & 18 & 34.7 \\
ZIP1 & JUMP (57\%) & 4.9 & 100 & 4 & $-$0.15 & 1 & \textbf{64.6} \\
ZIP2 & PASS (45\%) & 6.2 & 98 & 45 & $-$0.18 & 2 & 62.1 \\
\bottomrule
\multicolumn{8}{l}{\footnotesize Trade Time = mean timestep when trades occur (out of 100). Early\% = trades in first 30 steps.} \\
\multicolumn{8}{l}{\footnotesize SR = Spread Responsiveness. PIR = Price Improvement Rate (\% of quotes crossing spread).} \\
\end{tabular}
\end{table}


\subsection{Strategic Interaction in Heterogeneous Markets}
While self-play experiments establish baselines, placing strategies in a heterogeneous "mixed competition" market—with one buyer and one seller of each constrained type (ZIC, ZIC2, ZIP, ZIP2)—reveals insights into direct strategic interaction.

\subsubsection{The Value of Market Information in Varied Environments}
Our experiments suggest the value of market awareness is highly context-dependent. Table~\ref{tab:mixed_awareness} shows that ZIC2 (market-aware) outperforms ZIC (unaware) only under specific forms of resource scarcity. In the \textbf{SHRT} environment (scarce time), ZIC2 earns 42\% more profit. In the \textbf{BBBS} environment (scarce sellers), it earns 198\% more. However, in the volatile \textbf{RAN} environment, awareness is a liability, costing ZIC2 28\% of its profits relative to the "blind" ZIC, likely due to agents acting on stale information.

\begin{table}[H]
\centering
\caption{Market Awareness: ZIC2 vs ZIC1 Profit Gap Across Environments}
\label{tab:mixed_awareness}
\begin{tabular}{lrrrr}
\toprule
\textbf{Environment} & \textbf{ZIC1 Profit} & \textbf{ZIC2 Profit} & \textbf{Gap} & \textbf{Gap (\%)} \\
\midrule
BASE & 61 & 54 & $-$7 & $-$11 \\
BBBS & 11 & 34 & +22 & +198 \\
EQL & 61 & 59 & $-$2 & $-$3 \\
RAN & 849 & 613 & $-$236 & $-$28 \\
PER & 61 & 63 & +2 & +3 \\
SHRT & 33 & 46 & +14 & +42 \\
TOK & 8 & 13 & +5 & +62 \\
LAD & 62 & 50 & $-$12 & $-$20 \\
\bottomrule
\multicolumn{5}{l}{\footnotesize BSSS and SML omitted (ZIC strategies not viable). Context determines value of awareness.} \\
\end{tabular}
\end{table}


\subsubsection{Profit Mechanisms: High-Margin vs. High-Volume Strategies}
The mixed market also reveals that ZIC and ZIP achieve similar profitability through different mechanisms. In the BASE environment, ZIC and ZIP earn nearly identical total profits (61 and 64, respectively, see Table~\ref{tab:mixed_summary}). However, ZIC achieves the highest profit per trade (39.3) by occasionally capturing large-surplus "windfall" trades. In contrast, ZIP earns less per trade (30.0) but executes a higher volume of transactions. This reveals a classic strategic trade-off between a low-volume, high-margin strategy and a high-volume, consistent-margin strategy.

\subsubsection{The Cost of Heuristic Passivity}
A seemingly rational heuristic—waiting for a guaranteed price improvement—can be systematically punished. ZIP2, which combines learning with a rule that it must PASS if it cannot improve the current price, performs poorly in direct competition. Its PASS rate of 45\% in self-play is a quantitative signature of this patience. As shown in Table~\ref{tab:mixed_blindness}, this patient agent earns significantly less than its more aggressive ZIP counterpart across all ten environments, with a profit gap of 61\% in BASE. Its passivity causes it to miss trading opportunities captured by more aggressive agents.

\begin{table}[H]
\centering
\caption{Mixed Competition: Strategy Performance in BASE Environment}
\label{tab:mixed_summary}
\begin{tabular}{lrrrr}
\toprule
\textbf{Strategy} & \textbf{Profit} & \textbf{Rank} & \textbf{Win Rate (\%)} & \textbf{Profit/Trade} \\
\midrule
ZIC1 & 61 & 2.1 & 31 & 39.3 \\
ZIC2 & 54 & 2.4 & 26 & 27.0 \\
ZIP1 & \textbf{64} & \textbf{2.3} & \textbf{32} & 30.0 \\
ZIP2 & 25 & 3.2 & 10 & 11.8 \\
\bottomrule
\multicolumn{5}{l}{\footnotesize 4 buyers (1 each strategy) vs 4 sellers (1 each strategy). 100 rounds $\times$ 10 periods.} \\
\multicolumn{5}{l}{\footnotesize ZI excluded as loss-making distorts analysis. Bold = best in column.} \\
\end{tabular}
\end{table}

\begin{table}[H]
\centering
\caption{Institutional Blindness: ZIP2 vs ZIP1 Profit Gap Across Environments}
\label{tab:mixed_blindness}
\begin{tabular}{lrrrr}
\toprule
\textbf{Environment} & \textbf{ZIP1 Profit} & \textbf{ZIP2 Profit} & \textbf{Gap} & \textbf{Gap (\%)} \\
\midrule
BASE & 64 & 25 & $-$39 & $-$61 \\
BBBS & 30 & 9 & $-$22 & $-$71 \\
BSSS & 87 & 32 & $-$55 & $-$63 \\
EQL & 60 & 21 & $-$39 & $-$65 \\
RAN & 781 & 596 & $-$185 & $-$24 \\
PER & 78 & 23 & $-$55 & $-$70 \\
SHRT & 66 & 26 & $-$40 & $-$61 \\
TOK & 11 & 4 & $-$7 & $-$64 \\
SML & 59 & 12 & $-$47 & $-$80 \\
LAD & 65 & 21 & $-$44 & $-$68 \\
\bottomrule
\multicolumn{5}{l}{\footnotesize ZIP2's market-aware PASS rule systematically reduces profit by 24--80\%.} \\
\end{tabular}
\end{table}


\subsubsection{Market Sophistication and the Distribution of Surplus}
Finally, our results show a nuanced relationship between market sophistication and economic inequality. By the Gini coefficient, the unconstrained ZI market appears the most "equal" (0.26), while the adaptive ZIP market appears more unequal (0.43) (Table~\ref{tab:inequality}).

However, this metric can be misleading if not contextualized. In the ZI market, the bottom 50\% of traders achieve a negative profit share ($-$15.6\%). The apparent equality is an artifact of widespread surplus destruction. In contrast, in the ZIP market, the bottom 50\% of traders secure a positive profit share of 18\%. The Max/Mean profit ratio also falls from 52x in the ZI market to just 2x in the ZIP market. Sophistication provides a critical welfare floor by preventing the catastrophic losses that characterize the ZI market, leading to a more robust distribution of surplus among participants.

\begin{table}[H]
\centering
\caption{Selfplay Inequality Metrics (BASE Environment, 3 Seeds $\times$ 10 Rounds $\times$ 10 Periods)}
\label{tab:inequality}
\begin{tabular}{lrrrrr}
\toprule
\textbf{Metric} & \textbf{ZI} & \textbf{ZIC1} & \textbf{ZIC2} & \textbf{ZIP1} & \textbf{ZIP2} \\
\midrule
Gini & 0.26 & 0.39 & 0.41 & 0.43 & 0.44 \\
Max/Mean Ratio & 52.0 & 1.9 & 2.0 & 2.1 & 2.2 \\
Bottom-50\% Share & $-$15.6\% & 28.5\% & 23.3\% & 18.0\% & 16.5\% \\
Skewness & $+$0.03 & $+$0.20 & $+$0.20 & $+$0.12 & $+$0.14 \\
\bottomrule
\multicolumn{6}{l}{\footnotesize Gini = profit concentration (0 = equal, 1 = one agent takes all).} \\
\multicolumn{6}{l}{\footnotesize Max/Mean = highest earner relative to average. Bottom-50\% = share captured by lower half.} \\
\end{tabular}
\end{table}


\subsection{Summary of Findings}
Our replication and extension of these foundational experiments in the diverse Santa Fe environments lead to four primary conclusions:

\begin{enumerate}
    \item \textbf{Allocative efficiency is primarily an institutional outcome.} Our results are consistent with the Gode-Sunder hypothesis. The imposition of a simple budget constraint is the single most significant factor in achieving high allocative efficiency, a finding that holds across both simple and complex strategic environments.

    \item \textbf{Adaptive intelligence is essential for market coherence.} Our findings support a refined version of the Cliff-Bruten hypothesis. While not always necessary for high efficiency, adaptive intelligence is critical for the speed and certainty of convergence, allowing agents to overcome the coordination failures that plague non-adaptive strategies in scarce environments.

    \item \textbf{The value of simple heuristics is context-dependent.} In heterogeneous markets, the value of a given heuristic is not universal. The utility of market awareness depends on the market's structure, and seemingly rational "patient" behaviors can be systematically penalized in a competitive ecosystem.

    \item \textbf{Sophistication improves welfare by preventing market failure.} More sophisticated markets may not appear more "equal" by standard distributional metrics. However, they provide a crucial welfare floor by eliminating the catastrophic, surplus-destroying losses that characterize markets populated by unconstrained agents, ensuring a more stable and robust distribution of gains.
\end{enumerate}
